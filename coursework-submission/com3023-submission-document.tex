\documentclass[11pt,a4paper]{article}
\usepackage[utf8]{inputenc}
\usepackage[a4paper, margin=1in]{geometry}
\usepackage{graphicx}
\usepackage{minted}
\usepackage{caption}
\usepackage{float}


\begin{document}

\title{COM3023 Coursework Submission}
\author{Hallam Saunders (URN: 6788550)}
\date{\today}
\maketitle

\newpage

\tableofcontents
\newpage

\section{Demonstration Screenshots}

\subsection{12-into-1 Aggregation}
\begin{figure}[h!]
	\centering
	\includegraphics[width=\textwidth]{screenshots/12-into-1-aggregation.png}
	\caption{12-into-1 aggregation indicating little activity.}
	\label{fig:12-into-1-aggregation}
\end{figure}

\subsection{4-into-1 Aggregation}
\begin{figure}[h!]
	\centering
	\includegraphics[width=\textwidth]{screenshots/4-into-1-aggregation.png}
	\caption{4-into-1 aggregation indicating some activity.}
	\label{fig:4-into-1-aggregation}
\end{figure}

\subsection{No Aggregation}
\begin{figure}[h!]
	\centering
	\includegraphics[width=\textwidth]{screenshots/no-aggregation.png}
	\caption{No aggregation indicating high activity.}
	\label{fig:no-aggregation}
\end{figure}

\newpage

\section{Source Code}

\subsection{Pre-Demonstration}
Below is the source code for basic functionality required for the coursework, before the code demonstration task was implemented. Note that it is arranged such that functions can be viewed in their entirety and do not break over pages.

\sloppy
\noindent
\inputminted[linenos, breaklines, breakanywhere]{c}{code/code-p1.c}

\newpage
\inputminted[linenos, breaklines, breakanywhere, firstnumber=44]{c}{code/code-p2.c}

\newpage
\inputminted[linenos, breaklines, breakanywhere, firstnumber=74]{c}{code/code-p3.c}

\newpage
\inputminted[linenos, breaklines, breakanywhere, firstnumber=124]{c}{code/code-p4.c}

\newpage
\inputminted[linenos, breaklines, breakanywhere, firstnumber=142]{c}{code/code-p5.c}

\newpage
\inputminted[linenos, breaklines, breakanywhere, firstnumber=187]{c}{code/code-p6.c}

\fussy

\pagebreak

\subsection{Post-Demonstration}
Below is the source code after implementing the code demonstration task.

\end{document}
